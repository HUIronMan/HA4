\documentclass[a4paper]{report}
\usepackage{hyperref}
\usepackage{lastpage}
\usepackage{fancyhdr}
\usepackage{lineno}
\usepackage{listings}
\usepackage{german}
\usepackage[utf8]{inputenc}
\usepackage{amssymb}
\usepackage{graphicx}
\usepackage{pdflscape}
%\newcommand{\genasso}[2]{\begin{minipage}{0.7\textwidth}\begin{normalsize}\begin{flushleft}\textbf{{#1}}\end{flushleft}\end{normalsize}\vspace{-1cm}\begin{flushleft}\begin{small}{#2}\end{small}\end{flushleft}\end{minipage}\\\vspace{0.2cm}}
\pagenumbering{arabic}

\pagestyle{fancy} 
\newcommand{\frontmatter}{\clearpage \cfoot{\thepage\ }
\setcounter{page}{1}
\pagenumbering{Roman}}
\newcommand{\mainmatter}{\clearpage \lhead{\myAuth} \rhead{\myDate} \cfoot{} \rfoot{\thepage\ of \pageref{LastPage}}
\setcounter{page}{1}
\pagenumbering{arabic}}
\newcommand{\backmatter}{\clearpage \rfoot{\thepage\ }
\setcounter{page}{1}
\pagenumbering{alph}}


\newcommand{\makemytitlepage}{\begin{titlepage}
    \begin{center}
        \vspace*{0.8cm}
        
        \Huge
        \textbf{\myTitle}
        
        \vspace{1.5cm}
        
        \Large
        \myAuthor

        \vspace{1.8cm}

        %\begin{large}\textbf{Abstract:} \myAbstract \end{large}
        \includegraphics[width=6cm]{./IM.jpg}  
        
        \vfill
        
        \huge
        \myAsso
        
        \vspace{1.3cm}
        
        \Large

        %\myDate
        \today
        
    \end{center}
\end{titlepage}}
\newcommand{\myAuth}{Team: *Iron Man*\\B. Pohl, K. Trogant, R. Enseleit, D. Hebecker}
\newcommand{\myAuthor}{Birgit Pohl 574353 (MO. 9-11)\\Kevin Trogant 572451 (Mo. 15-17)\\Ronja Enseleit 572404 (Mo. 15-17)\\Dustin Hebecker 571271 (MO. 9-11)}
\newcommand{\myAsso}{Group: *Iron Man*}
\newcommand{\myDate}{\today}

%%%%%%%%%%%%%%%%%%%%%%%%%%%%%%%%
%%Change Title !!!!!!!!!!!!!!!!!
%%%%%%%%%%%%%%%%%%%%%%%%%%%%%%%%
\newcommand{\myTitle}{Exercise Sheet 4}

\begin{document}
\frontmatter
\makemytitlepage
\mainmatter

%%%%%%%%%%%%%%%%%%%%%%%%%%%%%%%%%%%%%%%%%%%%%%%%%%%%%%%%%%
%% Only modify below here  and change myTitle!!!!!!!!!!!!!
%%%%%%%%%%%%%%%%%%%%%%%%%%%%%%%%%%%%%%%%%%%%%%%%%%%%%%%%%%
\section*{Aufgabe 1}
Der Akzeptanztest zielt darauf ab beim Kunden Vertrauen in das Produkt zu erzeugen, während der Funktionstest überprüft ob das entwickelte Produkt den Anforderungen der Anforderungsdefinition entspricht. \\
Im Gegensatz zum Funktionstest wird der Akzeptanztest vom Kunden in der Umgebung durchgeführt, in der das Produkt eingesetzt werden soll. Da vor dem Akzeptanztest bereits durch den Funktionstest sichergestellt wurde, dass das Produkt der Anforderungsdefinition entspricht, sollte im Akzeptanztest keine Abweichung vom definierten Verhalten mehr entdeckt werden. Das heißt, im Akzeptanztest wird vor allem überprüft ob das Produkt auf die Weise benutzt werden kann, die sich der Kunde vorgestellt hat. \\
Schlägt der Funktionstest fehl, heißt das, dass das Produkt eine funktionale Anforderung nicht erfüllt (also zum Beispiel eine falsche Ausgabe liefert). Schlägt der Akzeptanztest fehl, heißt das, dass das Produkt vom Kunden nicht benutzt werden kann (zB. weil die Bedienung zu umständlich ist).

\newpage
\section*{Aufgabe 2}
\subsection*{Bildung von gültigen und ungültigen Äquivalenzklassen:}
\begin{tabular}{l|l|l}
    Eingabe 	& gültige Äquivalenzklassen 							& ungültige Äquivalenzklassen\\
    \hline
    1. String 	& 1) String aus Buchstaben und/oder Ziffern, der		& 11) Kein String\\
    			&    der gleiche String wie der 2. String ist 			& \\
    			& 2) String aus Buchstaben und/oder Ziffern, der		& \\
    			&    ein echter Substring von dem 2. String ist 		& \\
    			& 3) String aus Buchstaben und/oder Ziffern, der		& \\
    			&    kein Substring von dem 2. String ist				& \\
    			& 4) String mit Sonderzeichen, der						& \\
    			&    der gleiche String wie der 2. String ist 			& \\
    			& 5) String mit Sonderzeichen, der	 					& \\
    			&    ein echter Substring von dem 2. String ist 		& \\
    			& 6) String mit Sonderzeichen, der	 					& \\
    			&    kein Substring von dem 2. String ist				& \\
    			& 7) Leerer String										& \\
    			& 8) String mit Leerzeichen (vlt. auch Sonderzeichen),	& \\
    			&    der der gleiche String wie der 2. String ist 		& \\
    			& 9) String mit Leerzeichen (vlt. auch Sonderzeichen),	& \\
    			&    der ein echter Substring von dem 2. String ist 	& \\
    			& 10) String mit Leerzeichen (vlt. auch Sonderzeichen),& \\
    			&    der kein Substring von dem 2. String ist			& \\
    \hline
    2. String 	& 12) String aus Buchstaben und/oder Ziffern, der		& 22) Kein String\\
    			&    der gleiche String wie der 1. String ist 			& \\
    			& 13) String aus Buchstaben und/oder Ziffern, der		& \\
    			&    ein echter Substring von dem 1. String ist 		& \\
    			& 14) String aus Buchstaben und/oder Ziffern, der		& \\
    			&    kein Substring von dem 1. String ist				& \\
    			& 15) String mit Sonderzeichen, der					& \\
    			&    der gleiche String wie der 1. String ist 			& \\
    			& 16) String mit Sonderzeichen, der	 				& \\
    			&    ein echter Substring von dem 1. String ist 		& \\
    			& 17) String mit Sonderzeichen, der	 				& \\
    			&    kein Substring von dem 1. String ist				& \\
    			& 18) Leerer String										& \\
    			& 19) String mit Leerzeichen (vlt. auch Sonderzeichen),& \\
    			&    der der gleiche String wie der 1. String ist 		& \\
    			& 20) String mit Leerzeichen (vlt. auch Sonderzeichen),& \\
    			&    der ein echter Substring von dem 1. String ist 	& \\
    			& 21) String mit Leerzeichen (vlt. auch Sonderzeichen),& \\
    			&    der kein Substring von dem 1. String ist			& \\
\end{tabular} \\ \\
\subsection*{Aquivalenzklassentestschema:}
Strings sind hier der Übersichtlichkeit halber von Anführungszeichen umrandet.\\
\begin{tabular}{l|l|ll|l}
Testfall- 	& Getestete 		& Eingabe 1  	& Eingabe 2 	& Ergebnis/Kommentar\\
nummer		& Äquivalenzklassen	& 				&				&\\
\hline
0.			& 1, 12				& $\glqq$0d3r$\grqq$		& $\glqq$0d3r$\grqq$		& Kein Substring\\
1.			& 2, 14				& $\glqq$f5jk$\grqq$		& $\glqq$ghf5jk79$\grqq$	& 1. String ist ein Substring des 2.\\
2.			& 2, 14				& $\glqq$allo$\grqq$		& $\glqq$hallo$\grqq$		& 1. String ist ein Substring des 2.\\
3.			& 2, 14				& $\glqq$hall$\grqq$		& $\glqq$hallo$\grqq$		& 1. String ist ein Substring des 2.\\
4.			& 2, 14				& $\glqq$a$\grqq$			& $\glqq$hallo$\grqq$		& 1. String ist ein Substring des 2.\\
5.			& 2, 17				& $\glqq$allo$\grqq$		& $\glqq$\%hallo+$\grqq$	& 1. String ist ein Substring des 2.\\
6.			& 2, 21				& $\glqq$Hallo$\grqq$		& $\glqq$Hallo Welt!$\grqq$& 1. String ist ein Substring des 2.\\
7.			& 3, 13				& $\glqq$ghf5jk79$\grqq$	& $\glqq$hf5$\grqq$			& 2. String ist ein Substring des 1.\\
8.			& 3, 14				& $\glqq$ghf5jk79$\grqq$	& $\glqq$796g$\grqq$		& Kein Substring\\
9.			& 3, 17				& $\glqq$ghf5jk79$\grqq$	& $\glqq$hf5\&$\grqq$		& Kein Substring\\
10.			& 3, 18				& $\glqq$ghf5jk79$\grqq$	& $\glqq \grqq$				& Kein Substring\\
11.			& 3, 21				& $\glqq$ghf5jk79$\grqq$	& $\glqq$1 0h$\grqq$		& Kein Substring\\
12.			& 3, 22				& $\glqq$ghf5jk79$\grqq$	& 42						& Falsche Eingabe\\
13.			& 4, 15				& $\glqq$Hallo!$\grqq$		& $\glqq$Hallo!$\grqq$		& Kein Substring\\
14.			& 5, 17				& $\glqq$Hallo!$\grqq$		& $\glqq$Hallo!!!$\grqq$	& 1. String ist ein Substring des 2.\\
15.			& 5, 21				& $\glqq$!Hallo$\grqq$		& $\glqq$!Hallo Welt!$\grqq$& 1. String ist ein Substring des 2.\\
16.			& 6, 13				& $\glqq$Hallo!$\grqq$		& $\glqq$Hallo$\grqq$		& 2. String ist ein Substring des 1.\\
17.			& 6, 14				& $\glqq$Hallo!$\grqq$		& $\glqq$hallo0$\grqq$		& Kein Substring\\
18.			& 6, 16				& $\glqq$Hallo,d$\grqq$		& $\glqq$Hallo,$\grqq$		& 2. String ist ein Substring des 1.\\
19.			& 7, 17				& $\glqq \grqq$				& $\glqq$1\%$\grqq$			& Kein Substring\\
20.			& 7, 22				& $\glqq \grqq$				& 1.56						& Falsche Eingabe\\
21.			& 8, 19				& $\glqq$Hallo Welt!$\grqq$& $\glqq$Hallo Welt!$\grqq$& Kein Substring\\
22.			& 9, 21				& $\glqq$Hallo $\grqq$		& $\glqq$Hallo Welt!$\grqq$& 1. String ist ein Substring des 2.\\
23.			& 10, 20			& $\glqq$Hallo Welt!$\grqq$& $\glqq$Hallo $\grqq$		& 2. String ist ein Substring des 1.\\
24.			& 11, 14			& 127						& $\glqq$127$\grqq$			& Falsche Eingabe\\
25.			& 11, 22			& 1337						& 1337						& Falsche Eingabe\\
\end{tabular} \\ \\
%Das sind 25 Tests. An sich können alle Tests auch für den Funktionstest verwendet werden, denn wenn sie fehlschlagen, dann stimmt irgendetwas an dem Programm nicht.
Es ergeben sich 25 Tests. Diese Tests können alle für den Funktionstest verwendet werden, da sie die Funktionalität eines bestimmten Moduls testen. \\

\newpage
\section*{Aufgabe 3}
\subsection*{Äquivalenzklassen für den Modi ``Bearbeitung''}
\begin{tabular}{|l|l|l|}
\hline
    Eingabe 	& gültige Äquivalenzklassen 	& ungültige Äquivalenzklassen\\
    \hline
Stadt 		& 1) Noch nicht im System 	& 2) Schon im System \\
		& 				& 3) Leerer Wert \\
		& 				& 4) Numerischer Wert \\
\hline
Route Zeit 	& 5) $>0$ 			& 6) $<0$\\
		&  				& 7) $0$\\
		&  				& 8) Leer \\
		& 				& 9) Alphabetisch \\
\hline
Route Distanz 	& 10) $>0$ 			& 11) $<0$\\
		&  				& 12) $0$\\
		&  				& 13) Leer \\
		& 				& 14) Alphabetisch \\
\hline
Route Kosten 	& 15) Numerisch 		& 16) Alphabetisch \\
		&  				& 17) Leer \\
\hline
Route Stadt 1/2 & 18) 2 Unterschiedliche Städte aus dem Sytstem & 19) Leer \\
		&  						& 20) 2 Identische Städte\\
		&  						& 21) Stadt nicht im System \\
		&  						& 22) Numerischer Wert \\ 
\hline
\end{tabular}

\begin{landscape}
\subsection*{Tests für die Äquivalenzklassen des Modi ``Bearbeitung''}
\begin{tabular}{|l|l|l|l|}
\hline
Testfall- 	& Getestete 		& Eingaben 	& Ergebnis/Kommentar\\
nummer		& Äquivalenzk.		& 		&\\
\hline
1 		& 1			& Berlin	& Erste Stadt im System \\
2 		& 2			& Berlin	& Fehler Duplikat \\
3 		& 3			& \glqq\ \grqq  & Fehler leerer String\\
4 		& 4			& 1234		& Fehler ungültiger Wert \\
5 		& 5, 10, 15, 18		& T$=$1,S$=$100,V$=$10,C1$=$Köln,C2$=$Bonn & Erfolg, angenommen Die Städte sind im System\\
6 		& 6, 10, 15, 18		& T$=$-1,S$=$100,V$=$10,C1$=$Köln,C2$=$Bonn & Fehler negative Zeit\\
7 		& 7, 10, 15, 18		& T$=$0,S$=$100,V$=$10,C1$=$Köln,C2$=$Bonn & Fehler Instantane Reise\\
8 		& 8, 10, 15, 18		& T$=$,S$=$100,V$=$10,C1$=$Köln,C2$=$Bonn & Fehler Keine Zeit gegeben\\
9 		& 9, 10, 15, 18		& T$=$abc,S$=$100,V$=$10,C1$=$Köln,C2$=$Bonn & Fehler falscher Zeitwert\\
10 		& 5, 11, 15, 18		& T$=$1,S$=$-1,V$=$10,C1$=$Köln,C2$=$Bonn & Fehler negative Distanz\\
11 		& 5, 12, 15, 18		& T$=$1,S$=$0,V$=$10,C1$=$Köln,C2$=$Bonn & Fehler Städte haben keinen Abstand\\
12 		& 5, 13, 15, 18		& T$=$1,S$=$,V$=$10,C1$=$Köln,C2$=$Bonn & Fehler kein Städteabstand gegeben\\
13 		& 5, 14, 15, 18		& T$=$1,S$=$abc,V$=$10,C1$=$Köln,C2$=$Bonn & Fehler falscher Abstandswert\\
14 		& 5, 10, 16, 18		& T$=$1,S$=$100,V$=$abc,C1$=$Köln,C2$=$Bonn & Fehler falscher Preiswert\\
15 		& 5, 10, 17, 18		& T$=$1,S$=$100,V$=$,C1$=$Köln,C2$=$Bonn & Fehler kein Preis gegeben\\
16 		& 5, 10, 15, 19		& T$=$1,S$=$100,V$=$10,C1$=$,C2$=$Bonn & Fehler ein oder beide Städte nicht angegeben (beides testen)\\
17 		& 5, 10, 15, 20		& T$=$1,S$=$100,V$=$10,C1$=$Köln,C2$=$Köln & Fehler Start und Ziel identisch\\
18 		& 5, 10, 15, 21		& T$=$1,S$=$100,V$=$10,C1$=$Köln,C2$=$Moon & Fehler eine oder beide Städte nicht im System\\
19 		& 5, 10, 15, 22		& T$=$1,S$=$100,V$=$10,C1$=$Köln,C2$=$123 & Fehler falscher Städtewert\\ 
\hline
\end{tabular}\\
T$=$Zeit\\
S$=$Distanz\\
V$=$Kosten\\
C$=$Stadt\\
F$=$Fehler\\
\end{landscape}

\subsection*{Äquivalenzklassen für den Modi ``Abfrage''}
Alle folgenden Klassen gibt es für jede Abfrage (Zeit, Distanz, Kosten, Agony). Weiterhin wird angenommen die Abfragen sind so gestalltet, dass die Korrekte Antowrt bekannt ist bzw. das System expliziet für diese Abfrage gestaltet wurde.
\begin{tabular}{|l|l|l|}
\hline
    Eingabe 	& gültige Äquivalenzklassen 	& ungültige Äquivalenzklassen\\
    \hline
2 Städte 	& 1) Beide im System						& 4) Nicht im System \\
		& 2) 2 oder Mehr routen möglich.				& 5) Start und Ziel identisch \\
		& 3) Kürzeste Route führt über mehrere/alle 			& 6) Leere Abfrage \\
		&    anderen Städte im System 					& 7) Es existiert keien Verbindung \\
		&								& zwischen den Eingaben \\
\hline    
\end{tabular}


\subsection*{Tests für die Äquivalenzklassen des Modi ``Abfrage''}
Im folgenden wird angenommen es existiert ein System mit folgenden Städten:
\begin{itemize}
 \item Berlin
 \item Bonn
 \item Köln
 \item Düsseldorf
 \item New York
\end{itemize}
Als Agony wird der Einfachheit halber die Summe aus Zeit und Kosten verwendet. Das Produkt wäre jedoch realistischer.\\
Weiterhin werden folgende verbindungen angenommen:
\begin{enumerate}
 \item Berlin - Bonn , 5h, 500 km, 5 Euro, Agony 10 
 \item Berlin - Düsseldorf , 5h, 500 km, 5 Euro, Agony 10
 \item Berlin - Köln , 4h, 400 km, 4 Euro, Agony 8
 \item Köln - Bonn , 1h, 100 km, 1 Euro, Agony 2
 \item Köln - Düsseldorf , 10h, 1000 km, 10 Euro, Agony 20
 \end{enumerate}


\begin{tabular}{|l|l|l|l|}
\hline
Testfall- 	& Getestete 		& Eingaben 	& Ergebnis/Kommentar\\
nummer		& Äquivalenzk.		& 		&\\
\hline
1		& 1			& Köln - Bonn		& Take route 4 \\
2		& 2, (3)		& Berlin - Bonn		& Take route 1 or (3 and 4)\\
3		& 3			& Köln - Düsseldorf 	& Take route 3 and 2 \\
4		& 4			& Bielefeld - Berlin 	& Fehler Bielefeld existiert nicht\\
5		& 5			& Berlin - Berlin	& Fehler Start und Ziel identisch\\ 
6		& 6			&  - Berlin		& Fehler leerer Startort\\
7		& 7			& Berlin - New York	& Fehler es gibt keine Verbindung\\
\hline
\end{tabular}
Im allgemeinen sollte dieser Test noch etwas ausfühlerlicher sein und z.B. Strecken mit $1-n$ routen testen, welche sich bei einem schlechten Programierstiel utnerschiedlich verhalten könnten. Angenommen es wird sich an die gängigsten und sinnvollsten Techniken gehalten, sollte dieser Test jedoch ausreichen. Insbesondere zum test der Wegfindung (möglicherweise stochastischer Algorithmus) sollten dynamische verschiedene Modelle erstellt werden mit bekannten Lösungen und getestet werden. Ein sollcher Test sollte jedoch ohne Hintergrundwissen über die Code Struktur des Projektes erstellt werden (blind test). 


\end{document}
