\documentclass[a4paper]{report}
\usepackage{hyperref}
\usepackage{lastpage}
\usepackage{fancyhdr}
\usepackage{lineno}
\usepackage{listings}
\usepackage{german}
\usepackage[utf8]{inputenc}
\usepackage{amssymb}
\usepackage{graphicx}
%\newcommand{\genasso}[2]{\begin{minipage}{0.7\textwidth}\begin{normalsize}\begin{flushleft}\textbf{{#1}}\end{flushleft}\end{normalsize}\vspace{-1cm}\begin{flushleft}\begin{small}{#2}\end{small}\end{flushleft}\end{minipage}\\\vspace{0.2cm}}
\pagenumbering{arabic}

\pagestyle{fancy} 
\newcommand{\frontmatter}{\clearpage \cfoot{\thepage\ }
\setcounter{page}{1}
\pagenumbering{Roman}}
\newcommand{\mainmatter}{\clearpage \lhead{\myAuth} \rhead{\myDate} \cfoot{} \rfoot{\thepage\ of \pageref{LastPage}}
\setcounter{page}{1}
\pagenumbering{arabic}}
\newcommand{\backmatter}{\clearpage \rfoot{\thepage\ }
\setcounter{page}{1}
\pagenumbering{alph}}


\newcommand{\makemytitlepage}{\begin{titlepage}
    \begin{center}
        \vspace*{0.8cm}
        
        \Huge
        \textbf{\myTitle}
        
        \vspace{1.5cm}
        
        \Large
        \myAuthor

        \vspace{1.8cm}

        %\begin{large}\textbf{Abstract:} \myAbstract \end{large}
        \includegraphics[width=6cm]{./IM.jpg}  
        
        \vfill
        
        \huge
        \myAsso
        
        \vspace{1.3cm}
        
        \Large

        %\myDate
        \today
        
    \end{center}
\end{titlepage}}
\newcommand{\myAuth}{Team: *Iron Man*\\B. Pohl, K. Trogant, R. Enseleit, D. Hebecker}
\newcommand{\myAuthor}{Birgit Pohl 574353 (MO. 9-11)\\Kevin Trogant 572451 (Mo. 15-17)\\Ronja Enseleit 572404 (Mo. 15-17)\\Dustin Hebecker 571271 (MO. 9-11)}
\newcommand{\myAsso}{Group: *Iron Man*}
\newcommand{\myDate}{\today}

%%%%%%%%%%%%%%%%%%%%%%%%%%%%%%%%
%%Change Title !!!!!!!!!!!!!!!!!
%%%%%%%%%%%%%%%%%%%%%%%%%%%%%%%%
\newcommand{\myTitle}{Exercise Sheet 4}

\begin{document}
\frontmatter
\makemytitlepage
\mainmatter

%%%%%%%%%%%%%%%%%%%%%%%%%%%%%%%%%%%%%%%%%%%%%%%%%%%%%%%%%%
%% Only modify below here  and change myTitle!!!!!!!!!!!!!
%%%%%%%%%%%%%%%%%%%%%%%%%%%%%%%%%%%%%%%%%%%%%%%%%%%%%%%%%%
\section*{Aufgabe 1}
Der Akzeptanztest zielt darauf ab beim Kunden Vertrauen in das Produkt zu erzeugen, während der Funktionstest überprüft ob das entwickelte Produkt den Anforderungen der Anforderungsdefinition entspricht. \\
Im Gegensatz zum Funktionstest wird der Akzeptanztest vom Kunden in der Umgebung durchgeführt, in der das Produkt eingesetzt werden soll. Da vor dem Akzeptanztest bereits durch den Funktionstest sichergestellt wurde, dass das Produkt der Anforderungsdefinition entspricht, sollte im Akzeptanztest keine Abweichung vom definierten Verhalten mehr entdeckt werden. Das heißt, im Akzeptanztest wird vor allem überprüft ob das Produkt auf die Weise benutzt werden kann, die sich der Kunde vorgestellt hat. \\
Schlägt der Funktionstest fehl, heißt das, dass das Produkt eine funktionale Anforderung nicht erfüllt (also zum Beispiel eine falsche Ausgabe liefert). Schlägt der Akzeptanztest fehl, heißt das, dass das Produkt vom Kunden nicht benutzt werden kann (zB. weil die Bedienung zu umständlich ist).

\newpage
\section*{Aufgabe 2}

\newpage
\section*{Aufgabe 3}
\subsection*{Äquivalenzklassen für den Modi ``Bearbeitung''}
Gültige Äquivalenzklassen:
Zeit positiv
Distanz positiv
Kosten numerisch
2 unterschiedliche städte


Ungültige Äquivalenzklassen:
negative Zeit/Distanz
alphabetische Zeit/Distanz/kosten
ein wert nicht angegeben
2 identische städte


\subsection*{Äquivalenzklassen für den Modi ``Abfrage''}
Gültige Äquivalenzklassen:
wert bezüglich (zeit/distanz/kosten/agony) bekannt
zwei equivalente routen möglich
kürzeste route führt über alle anderen/eine weitere städte

Ungültige Äquivalenzklassen:
Start und zielort identisch
Start oder zielort unbekannt
es gibt keine verbindung zwischen start und zielort

%\begin{lstlisting}
%Put your code here.
%\end{lstlisting}



\end{document}
