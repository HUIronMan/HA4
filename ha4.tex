\documentclass[a4paper]{report}
\usepackage{hyperref}
\usepackage{lastpage}
\usepackage{fancyhdr}
\usepackage{lineno}
\usepackage{listings}
\usepackage{german}
\usepackage[utf8]{inputenc}
\usepackage{amssymb}
\usepackage{graphicx}
%\newcommand{\genasso}[2]{\begin{minipage}{0.7\textwidth}\begin{normalsize}\begin{flushleft}\textbf{{#1}}\end{flushleft}\end{normalsize}\vspace{-1cm}\begin{flushleft}\begin{small}{#2}\end{small}\end{flushleft}\end{minipage}\\\vspace{0.2cm}}
\pagenumbering{arabic}

\pagestyle{fancy} 
\newcommand{\frontmatter}{\clearpage \cfoot{\thepage\ }
\setcounter{page}{1}
\pagenumbering{Roman}}
\newcommand{\mainmatter}{\clearpage \lhead{\myAuth} \rhead{\myDate} \cfoot{} \rfoot{\thepage\ of \pageref{LastPage}}
\setcounter{page}{1}
\pagenumbering{arabic}}
\newcommand{\backmatter}{\clearpage \rfoot{\thepage\ }
\setcounter{page}{1}
\pagenumbering{alph}}


\newcommand{\makemytitlepage}{\begin{titlepage}
    \begin{center}
        \vspace*{0.8cm}
        
        \Huge
        \textbf{\myTitle}
        
        \vspace{1.5cm}
        
        \Large
        \myAuthor

        \vspace{1.8cm}

        %\begin{large}\textbf{Abstract:} \myAbstract \end{large}
        \includegraphics[width=6cm]{./IM.jpg}  
        
        \vfill
        
        \huge
        \myAsso
        
        \vspace{1.3cm}
        
        \Large

        %\myDate
        \today
        
    \end{center}
\end{titlepage}}
\newcommand{\myAuth}{Team: *Iron Man*\\B. Pohl, K. Trogant, R. Enseleit, D. Hebecker}
\newcommand{\myAuthor}{Birgit Pohl 574353 (MO. 9-11)\\Kevin Trogant 572451 (Mo. 15-17)\\Ronja Enseleit 572404 (Mo. 15-17)\\Dustin Hebecker 571271 (MO. 9-11)}
\newcommand{\myAsso}{Group: *Iron Man*}
\newcommand{\myDate}{\today}

%%%%%%%%%%%%%%%%%%%%%%%%%%%%%%%%
%%Change Title !!!!!!!!!!!!!!!!!
%%%%%%%%%%%%%%%%%%%%%%%%%%%%%%%%
\newcommand{\myTitle}{Exercise Sheet 4}

\begin{document}
\frontmatter
\makemytitlepage
\mainmatter

%%%%%%%%%%%%%%%%%%%%%%%%%%%%%%%%%%%%%%%%%%%%%%%%%%%%%%%%%%
%% Only modify below here  and change myTitle!!!!!!!!!!!!!
%%%%%%%%%%%%%%%%%%%%%%%%%%%%%%%%%%%%%%%%%%%%%%%%%%%%%%%%%%
\section*{Aufgabe 1}
Der Akzeptanztest zielt darauf ab beim Kunden Vertrauen in das Produkt zu erzeugen, während der Funktionstest überprüft ob das entwickelte Produkt den Anforderungen der Anforderungsdefinition entspricht. \\
Im Gegensatz zum Funktionstest wird der Akzeptanztest vom Kunden in der Umgebung durchgeführt, in der das Produkt eingesetzt werden soll. Da vor dem Akzeptanztest bereits durch den Funktionstest sichergestellt wurde, dass das Produkt der Anforderungsdefinition entspricht, sollte im Akzeptanztest keine Abweichung vom definierten Verhalten mehr entdeckt werden. Das heißt, im Akzeptanztest wird vor allem überprüft ob das Produkt auf die Weise benutzt werden kann, die sich der Kunde vorgestellt hat. \\
Schlägt der Funktionstest fehl, heißt das, dass das Produkt eine funktionale Anforderung nicht erfüllt (also zum Beispiel eine falsche Ausgabe liefert). Schlägt der Akzeptanztest fehl, heißt das, dass das Produkt vom Kunden nicht benutzt werden kann (zB. weil die Bedienung zu umständlich ist).

\newpage
\section*{Aufgabe 2}
\subsection*{Bildung von gültigen und ungültigen Äquivalenzklassen:}
\begin{tabular}{l|l|l}
    Eingabe 	& gültige Äquivalenzklassen 							& ungültige Äquivalenzklassen\\
    \hline
    1. String 	& 1) String aus Buchstaben und/oder Ziffern, der		& 11) Kein String\\
    			&    der gleiche String wie der 2. String ist 			& \\
    			& 2) String aus Buchstaben und/oder Ziffern, der		& \\
    			&    ein echter Substring von dem 2. String ist 		& \\
    			& 3) String aus Buchstaben und/oder Ziffern, der		& \\
    			&    kein Substring von dem 2. String ist				& \\
    			& 4) String mit Sonderzeichen, der						& \\
    			&    der gleiche String wie der 2. String ist 			& \\
    			& 5) String mit Sonderzeichen, der	 					& \\
    			&    ein echter Substring von dem 2. String ist 		& \\
    			& 6) String mit Sonderzeichen, der	 					& \\
    			&    kein Substring von dem 2. String ist				& \\
    			& 7) Leerer String										& \\
    			& 8) String mit Leerzeichen (vlt. auch Sonderzeichen),	& \\
    			&    der der gleiche String wie der 2. String ist 		& \\
    			& 9) String mit Leerzeichen (vlt. auch Sonderzeichen),	& \\
    			&    der ein echter Substring von dem 2. String ist 	& \\
    			& 10) String mit Leerzeichen (vlt. auch Sonderzeichen),& \\
    			&    der kein Substring von dem 2. String ist			& \\
    \hline
    2. String 	& 12) String aus Buchstaben und/oder Ziffern, der		& 22) Kein String\\
    			&    der gleiche String wie der 1. String ist 			& \\
    			& 13) String aus Buchstaben und/oder Ziffern, der		& \\
    			&    ein echter Substring von dem 1. String ist 		& \\
    			& 14) String aus Buchstaben und/oder Ziffern, der		& \\
    			&    kein Substring von dem 1. String ist				& \\
    			& 15) String mit Sonderzeichen, der					& \\
    			&    der gleiche String wie der 1. String ist 			& \\
    			& 16) String mit Sonderzeichen, der	 				& \\
    			&    ein echter Substring von dem 1. String ist 		& \\
    			& 17) String mit Sonderzeichen, der	 				& \\
    			&    kein Substring von dem 1. String ist				& \\
    			& 18) Leerer String										& \\
    			& 19) String mit Leerzeichen (vlt. auch Sonderzeichen),& \\
    			&    der der gleiche String wie der 1. String ist 		& \\
    			& 20) String mit Leerzeichen (vlt. auch Sonderzeichen),& \\
    			&    der ein echter Substring von dem 1. String ist 	& \\
    			& 21) String mit Leerzeichen (vlt. auch Sonderzeichen),& \\
    			&    der kein Substring von dem 1. String ist			& \\
\end{tabular} \\ \\
\subsection*{Aquivalenzklassentestschema:}
Strings sind hier der Übersichtlichkeit halber von Anführungszeichen umrandet.\\
\begin{tabular}{l|l|ll|l}
Testfall- 	& Getestete 		& Eingabe 1  	& Eingabe 2 	& Ergebnis/Kommentar\\
nummer		& Äquivalenzklassen	& 				&				&\\
\hline
0.			& 1, 12				& $\glqq$0d3r$\grqq$		& $\glqq$0d3r$\grqq$		& Kein Substring\\
1.			& 2, 14				& $\glqq$f5jk$\grqq$		& $\glqq$ghf5jk79$\grqq$	& 1. String ist ein Substring des 2.\\
2.			& 2, 14				& $\glqq$allo$\grqq$		& $\glqq$hallo$\grqq$		& 1. String ist ein Substring des 2.\\
3.			& 2, 14				& $\glqq$hall$\grqq$		& $\glqq$hallo$\grqq$		& 1. String ist ein Substring des 2.\\
4.			& 2, 14				& $\glqq$a$\grqq$			& $\glqq$hallo$\grqq$		& 1. String ist ein Substring des 2.\\
5.			& 2, 17				& $\glqq$allo$\grqq$		& $\glqq$\%hallo+$\grqq$	& 1. String ist ein Substring des 2.\\
6.			& 2, 21				& $\glqq$Hallo$\grqq$		& $\glqq$Hallo Welt!$\grqq$& 1. String ist ein Substring des 2.\\
7.			& 3, 13				& $\glqq$ghf5jk79$\grqq$	& $\glqq$hf5$\grqq$			& 2. String ist ein Substring des 1.\\
8.			& 3, 14				& $\glqq$ghf5jk79$\grqq$	& $\glqq$796g$\grqq$		& Kein Substring\\
9.			& 3, 17				& $\glqq$ghf5jk79$\grqq$	& $\glqq$hf5\&$\grqq$		& Kein Substring\\
10.			& 3, 18				& $\glqq$ghf5jk79$\grqq$	& $\glqq \grqq$				& Kein Substring\\
11.			& 3, 21				& $\glqq$ghf5jk79$\grqq$	& $\glqq$1 0h$\grqq$		& Kein Substring\\
12.			& 3, 22				& $\glqq$ghf5jk79$\grqq$	& 42						& Falsche Eingabe\\
13.			& 4, 15				& $\glqq$Hallo!$\grqq$		& $\glqq$Hallo!$\grqq$		& Kein Substring\\
14.			& 5, 17				& $\glqq$Hallo!$\grqq$		& $\glqq$Hallo!!!$\grqq$	& 1. String ist ein Substring des 2.\\
15.			& 5, 21				& $\glqq$!Hallo$\grqq$		& $\glqq$!Hallo Welt!$\grqq$& 1. String ist ein Substring des 2.\\
16.			& 6, 13				& $\glqq$Hallo!$\grqq$		& $\glqq$Hallo$\grqq$		& 2. String ist ein Substring des 1.\\
17.			& 6, 14				& $\glqq$Hallo!$\grqq$		& $\glqq$hallo0$\grqq$		& Kein Substring\\
18.			& 6, 16				& $\glqq$Hallo,d$\grqq$		& $\glqq$Hallo,$\grqq$		& 2. String ist ein Substring des 1.\\
19.			& 7, 17				& $\glqq \grqq$				& $\glqq$1\%$\grqq$			& Kein Substring\\
20.			& 7, 22				& $\glqq \grqq$				& 1.56						& Falsche Eingabe\\
21.			& 8, 19				& $\glqq$Hallo Welt!$\grqq$& $\glqq$Hallo Welt!$\grqq$& Kein Substring\\
22.			& 9, 21				& $\glqq$Hallo $\grqq$		& $\glqq$Hallo Welt!$\grqq$& 1. String ist ein Substring des 2.\\
23.			& 10, 20			& $\glqq$Hallo Welt!$\grqq$& $\glqq$Hallo $\grqq$		& 2. String ist ein Substring des 1.\\
24.			& 11, 14			& 127						& $\glqq$127$\grqq$			& Falsche Eingabe\\
25.			& 11, 22			& 1337						& 1337						& Falsche Eingabe\\
\end{tabular} \\ \\
%Das sind 25 Tests. An sich können alle Tests auch für den Funktionstest verwendet werden, denn wenn sie fehlschlagen, dann stimmt irgendetwas an dem Programm nicht.
Es ergeben sich 25 Tests. Diese Tests können alle für den Funktionstest verwendet werden, da sie die Funktionalität eines bestimmten Moduls testen. \\

\newpage
\section*{Aufgabe 3}
\subsection*{Äquivalenzklassen für den Modi ``Bearbeitung''}
\begin{tabular}{|l|l|l|}
    Eingabe 	& gültige Äquivalenzklassen 	& ungültige Äquivalenzklassen\\
    \hline
Stadt 		& 1) Noch nicht im System 	& 2) Schon im System \\
		& 				& 3) Leerer Wert \\
		& 				& 4) Numerischer Wert \\
\hline
Route Zeit 	& 5) $>0$ 			& 6) $<0$\\
		&  				& 7) $0$\\
		&  				& 8) Leer \\
		& 				& 9) Alphabetisch \\
\hline
Route Distanz 	& 10) $>0$ 			& 11) $<0$\\
		&  				& 12) $0$\\
		&  				& 13) Leer \\
		& 				& 14) Alphabetisch \\
\hline
Route Kosten 	& 15) Numerisch 		& 16) Alphabetisch \\
		&  				& 17) Leer \\
\hline
Route Stadt 1/2 & 18) 2 Unterschiedliche Städte aus dem Sytstem & 19) Leer \\
		&  						& 20) 2 Identische Städte\\
		&  						& 21) Stadt nicht im System \\
		&  						& 22) Numerischer Wert \\ 
\hline
\end{tabular}


\subsection*{Tests für die Äquivalenzklassen des Modi ``Bearbeitung''}
\begin{tabular}{l|l|l|l}
Testfall- 	& Getestete 		& Eingaben 	& Ergebnis/Kommentar\\
nummer		& Äquivalenzklassen	& 		&\\
\hline
1 		& 1			& Berlin	& Erste Stadt im System \\
2 		& 2			& Berlin	& Fehlermeldung Duplikat \\
3 		& 3			& \glqq\ \grqq  & Fehlermeldung leerer String\\
4 		& 4			& 1234		& Fehlermeldung ungültiger Wert \\
5 		& 5, 10, 15, 18		& Zeit$=$1h,Distanz$=$100km,Kosten$=$10Euro,Stadt1$=$Düsseldorf,Stadt2$=$Bonn & Erfolg, angenommen Die Städte sind im System\\
6 		& 6, 10, 15, 18		& Zeit$=$-1h,Distanz$=$100km,Kosten$=$10Euro,Stadt1$=$Düsseldorf,Stadt2$=$Bonn & Fehlermeldung negative Zeit\\
7 		& 7, 10, 15, 18		& Zeit$=$0h,Distanz$=$100km,Kosten$=$10Euro,Stadt1$=$Düsseldorf,Stadt2$=$Bonn & Fehlermeldung Instantane Reise\\
8 		& 8, 10, 15, 18		& Zeit$=$,Distanz$=$100km,Kosten$=$10Euro,Stadt1$=$Düsseldorf,Stadt2$=$Bonn & Fehlermeldung Keine Zeit gegeben\\
9 		& 9, 10, 15, 18		& Zeit$=$abc,Distanz$=$100km,Kosten$=$10Euro,Stadt1$=$Düsseldorf,Stadt2$=$Bonn & Fehlermeldung falscher Zeitwert\\
10 		& 5, 11, 15, 18		& Zeit$=$1h,Distanz$=$-1km,Kosten$=$10Euro,Stadt1$=$Düsseldorf,Stadt2$=$Bonn & Fehlermeldung negative Distanz\\
11 		& 5, 12, 15, 18		& Zeit$=$1h,Distanz$=$0km,Kosten$=$10Euro,Stadt1$=$Düsseldorf,Stadt2$=$Bonn & Fehlermeldung Städte haben keinen Abstand\\
12 		& 5, 13, 15, 18		& Zeit$=$1h,Distanz$=$,Kosten$=$10Euro,Stadt1$=$Düsseldorf,Stadt2$=$Bonn & Fehlermeldung kein Städteabstand gegeben\\
13 		& 5, 14, 15, 18		& Zeit$=$1h,Distanz$=$abc,Kosten$=$10Euro,Stadt1$=$Düsseldorf,Stadt2$=$Bonn & Fehlermeldung falscher Abstandswert\\
14 		& 5, 10, 16, 18		& Zeit$=$1h,Distanz$=$100km,Kosten$=$abc,Stadt1$=$Düsseldorf,Stadt2$=$Bonn & Fehlermeldung falscher Preiswert\\
15 		& 5, 10, 17, 18		& Zeit$=$1h,Distanz$=$100km,Kosten$=$,Stadt1$=$Düsseldorf,Stadt2$=$Bonn & Fehlermeldung kein Preis gegeben\\
16 		& 5, 10, 15, 19		& Zeit$=$1h,Distanz$=$100km,Kosten$=$10Euro,Stadt1$=$,Stadt2$=$Bonn & Fehlermeldung ein oder beide Städte nicht angegeben (beides testen)\\
17 		& 5, 10, 15, 20		& Zeit$=$1h,Distanz$=$100km,Kosten$=$10Euro,Stadt1$=$Düsseldorf,Stadt2$=$Düsseldorf & Fehlermeldung Start und Ziel identisch\\
18 		& 5, 10, 15, 21		& Zeit$=$1h,Distanz$=$100km,Kosten$=$10Euro,Stadt1$=$Düsseldorf,Stadt2$=$Moon & Fehlermeldung eine oder beide Städte nicht im System\\
19 		& 5, 10, 15, 22		& Zeit$=$1h,Distanz$=$100km,Kosten$=$10Euro,Stadt1$=$Düsseldorf,Stadt2$=$123 & Fehlermeldung falscher Städtewert\\
\end{tabular}

\subsection*{Äquivalenzklassen für den Modi ``Abfrage''}
\begin{tabular}{|l|l|l|}
    Eingabe 	& gültige Äquivalenzklassen 	& ungültige Äquivalenzklassen\\
    \hline
wert bezüglich (zeit/distanz/kosten/agony) bekannt
zwei equivalente routen möglich
kürzeste route führt über alle anderen/eine weitere städte

Ungültige Äquivalenzklassen:
Start und zielort identisch
Start oder zielort unbekannt
es gibt keine verbindung zwischen start und zielort

\subsection*{Tests für die Äquivalenzklassen des Modi ``Abfrage''}

%\begin{lstlisting}
%Put your code here.
%\end{lstlisting}



\end{document}
